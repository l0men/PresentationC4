%%%%%%%%%%%%%%%%%%%%%%%%%%%%%%%%%%%%%%%%%%%%%%%%%%%%%%%%%%%%%%%%%%
%                                                                %
%                                                                %
%                       PACKAGE                                  %
%                                                                %
%                                                                %
%%%%%%%%%%%%%%%%%%%%%%%%%%%%%%%%%%%%%%%%%%%%%%%%%%%%%%%%%%%%%%%%%%

%  Basique :

\usepackage[utf8]{inputenc}
\usepackage[T1]{fontenc}
\usepackage[french]{babel} 
\usepackage{hyperref}                           %référence
%\usepackage[margin=1,25cm]{geometry}           %Géométrie

% Mathématiques :
\usepackage{cancel}                             %rayer 
\usepackage{amsthm,amsmath,amsfonts,amssymb,dsfont}    %maths et joli....
\usepackage{stmaryrd}                           %crochet entre autre

% Mise en forme : 

\usepackage[x11names]{xcolor}                   % inclus les couleurs
%\usepackage{enumitem}                           % Énumération avec label
\usepackage{multicol}                           % Colonne
\usepackage{graphicx}                           % Pour le bigcdot notamment
\usepackage{tcolorbox}                          % Box
\setlength{\fboxrule}{3pt}
\setlength{\fboxsep}{1em}




% Figure et Image :

\usepackage{wrapfig}                            %pour inclure des figures dans le texte  

%Lettres grecques :
\usepackage{savesym}
\usepackage{amsmath, amssymb}
\savesymbol{iint}
\usepackage{txfonts,, newtxmath}
\restoresymbol{TXF}{iint}
%\usepackage{mathptmx} %uses times fontv

%Beamer
\usepackage{booktabs, comment} 
\usepackage[absolute, overlay]{textpos}
\usepackage{pgfpages}
\usepackage[font=footnotesize]{caption}
\useoutertheme{infolines} 
\usepackage{csquotes}
\usepackage{textpos}
\usepackage{tikz}

%%%%%%%%%%%%%%%%%%%%%%%%%%%%%%%%%%%%%%%%%%%%%%%%%%%%%%%%%%%%%%%%%%
%                                                                %
%                                                                %
%                       Théorème                                 %
%                                                                %
%                                                                %
%%%%%%%%%%%%%%%%%%%%%%%%%%%%%%%%%%%%%%%%%%%%%%%%%%%%%%%%%%%%%%%%%%

\newtheorem*{remark}{Remark}

\theoremstyle{plain}% default
\newtheorem{Th}{Théorème}[section]
\newtheorem{Lem}[Th]{Lemme}
\newtheorem{Prop}[Th]{Proposition}
\newtheorem*{Prop*}{Proposition}
\newtheorem*{Cor}{Corollaire}
\newtheorem*{Obj}{Objection}

\theoremstyle{definition}
\newtheorem{Df}{Définition}[section]
\newtheorem*{Df*}{Définition}
\newtheorem{Dfs}[Df]{Définitions}%[section]
\newtheorem{conj}{conjecture}[section]
\newtheorem{Ex}{Exemple}[section]
\newtheorem*{Ex*}{Exemple}
\newtheorem{Exs}[Ex]{Exemples}%[section]

\theoremstyle{remark}
\newtheorem*{NB}{Remarque}
\newtheorem*{NBs}{Remarques}
\newtheorem*{Gen}{Généralisation}
\newtheorem*{nota}{Notation}
\newtheorem*{Rap}{Rappel}


\renewcommand{\proofname}{\textsc{Preuve}}

\newtheoremstyle{exostyle}{\topsep}{\topsep}{}{}{\bfseries}{.}{ }{\thmname{#1}\thmnumber{ #2}\thmnote{. \normalfont{\textit{#3}} }}

\theoremstyle{exostyle}
\newtheorem{exercice}{Exercice} 

\newtheoremstyle{exostyle}{\topsep}{\topsep}{}{}{\bfseries}{.}{ }{\thmname{#1}\thmnumber{ #2}\thmnote{. \normalfont{\textit{#3}}}}

\theoremstyle{exostyle}
\newtheorem*{preuve}{Preuve} 

\newenvironment{questions}{\begin{enumerate}}{\end{enumerate}}
%\newenvironment{sousquestions}{\begin{enumerate}}{\end{enumerate}}


%%%%%%%%%%%%%%%%%%%%%%%%%%%%%%%%%%%%%%%%%%%%%%%%%%%%%%%%%%%%%%%%%%
%                                                                %
%                                                                %
%                       Code                                     %
%                                                                %
%                                                                %
%%%%%%%%%%%%%%%%%%%%%%%%%%%%%%%%%%%%%%%%%%%%%%%%%%%%%%%%%%%%%%%%%%

\usepackage{color}
\usepackage{listings}
\definecolor{mygreen}{rgb}{0,0.6,0}
\definecolor{mygray}{rgb}{0.5,0.5,0.5}
\definecolor{mymauve}{rgb}{0.58,0,0.82}
\definecolor{myorange}{rgb}{0.855,0.576,0.027}
\lstset{
language=Octave,
frame=single,   
morecomment = [l][\itshape\color{blue}]{\%},
keywordstyle=\color{blue},
commentstyle=\color{mygreen},
breakatwhitespace=false,         
breaklines=true,  
numbers=left,
numbersep=5pt,
numberstyle=\tiny\color{mygray}, 
showstringspaces=false,
showtabs=false,                  
tabsize=2,
stringstyle=\color{myorange},
title=\lstname,
literate=
{+}{{{\color{red}+}}}1
{-}{{{\color{red}-}}}1
{*}{{{\color{red}*}}}1
{,}{{{\color{red},}}}1
{=}{{{\color{red}=}}}1
{(}{{{\color{black}(}}}1
{)}{{{\color{red})}}}1
{;}{{{\color{red};}}}1
{:}{{{\color{red}:}}}1
{[}{{{\color{red}[}}}1
{]}{{{\color{red}]}}}1
{>}{{{\color{red}>}}}1
}

%%%%%%%%%%%%%%%%%%%%%%%%%%%%%%%%%%%%%%%%%%%%%%%%%%%%%%%%%%%%%%%%%%
%                                                                %
%                                                                %
%                       Commandes                                %
%                                                                %
%                                                                %
%%%%%%%%%%%%%%%%%%%%%%%%%%%%%%%%%%%%%%%%%%%%%%%%%%%%%%%%%%%%%%%%%%

\newcommand{\N}{\mathbb{N}}
\newcommand{\Z}{\mathbb{Z}}
\newcommand{\R}{\mathbb{R}}
\newcommand{\C}{\mathbb{C}}
\newcommand{\D}{\mathcal D}
\newcommand{\Cl}{\mathcal C}
\newcommand{\Sm}{S^m_{0,1}}
\newcommand{\et}{\text{ ~et~ }}
\newcommand{\ou}{\text{ ~ou~ }}
\newcommand{\espace}{\vspace*{1cm}}
\everymath{\displaystyle}
\renewcommand{\Im}{\text{Im}}
\renewcommand{\Re}{\text{Re}}
\makeatletter
\newcommand*\bigcdot{\mathpalette\bigcdot@{.4}}
\newcommand*\bigcdot@[2]{\mathbin{\vcenter{\hbox{\scalebox{#2}{$\m@th#1\bullet$}}}}}
\makeatother

\newcommand{\framewithtitle}[1]{
\begin{frame}[c]{  }
\Huge{
\begin{center}
    \begin{tcolorbox}[arc=1ex, colback=myuniversity, colframe=myuniversity, left=3pt, right=3pt, top=3pt, bottom=2pt]
    \emph{
    \espace
    \begin{center}
        \textcolor{white}{#1}
    \end{center}
    \espace}
    \end{tcolorbox}
\end{center}}
\end{frame}
}

%%%%%%%%%%%%%%%%%%%%%%%%%%%%%%%%%%%%%%%%%%%%%%%%%%%%%%%%%%%%%%%%%%
%                                                                %
%                                                                %
%                       Beamer                                   %
%                                                                %
%                                                                %
%%%%%%%%%%%%%%%%%%%%%%%%%%%%%%%%%%%%%%%%%%%%%%%%%%%%%%%%%%%%%%%%%%



%% lien vers celui de janvier : https://fr.overleaf.com/8125717386bqbmvyhtchbt

\definecolor{whitetxt}{RGB}{255,255,255}
\definecolor{bluewhite}{RGB}{172,124,148}

\setbeamercolor{title in head/foot}{bg=bluewhite, fg=whitetxt}
\setbeamercolor{author in head/foot}{bg=myuniversity}
\setbeamertemplate{page number in head/foot}{}

\usetheme{Madrid}
\definecolor{myuniversity}{RGB}{116,28,76}
\usecolortheme[named=myuniversity]{structure}
\usepackage{tikz}


\addtobeamertemplate{navigation symbols}{}{%
    \usebeamerfont{footline}%
    \usebeamercolor[fg]{footline}%
    \hspace{1em}%
    \insertframenumber/\inserttotalframenumber
}

\logo{\includegraphics[height=1.0cm]{petitlogo.png}~%
}

% Couleur des puces
\setbeamercolor{itemize item}{fg=myuniversity}
\setbeamercolor{itemize subitem}{fg=myuniversity}
\setbeamercolor{itemize subsubitem}{fg=myuniversity}

% Forme des puces
\setbeamertemplate{itemize item}{\color{myuniversity}\large\textbullet}
\setbeamertemplate{itemize subitem}{\color{myuniversity}\small\textbullet}
\setbeamertemplate{itemize subsubitem}{\color{myuniversity}\tiny\textbullet}

\usepackage{pifont}
\newcommand{\fullstar}{\ding{72}}  % étoile pleine
\newcommand{\emptystar}{\ding{73}} % étoile vide
