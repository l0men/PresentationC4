\framewithtitle{Architecture du Système et Méthodologie}

\subsection{L'architecture du modèle Autoencodeur}

\begin{frame}{Architecture du modèle Autoencodeur}
   
\begin{columns}
\column{0.5\textwidth}
\vspace{20 pt}
\begin{block}{Extraction de features}
    Image $32*32*3 \Longrightarrow$ Espace latent
    \begin{itemize}
        \item Convolution 2D
        \item Batch Normalization 2D
        \item ReLU
    \end{itemize}
\end{block}

\column{0.5\textwidth}
\begin{figure}
    \centering
    \includegraphics[width=0.8\linewidth]{img/encodeur.png}
    \caption{Encodeur convolutionel}
    \label{fig:encodeur}
\end{figure}
\end{columns}
\end{frame}

\begin{frame}{Architecture du modèle Autoencodeur}
   
\begin{columns}
\column{0.5\textwidth}
\vspace{20 pt}
\begin{block}{Reconstruction de l'image}
    Espace latent $\Longrightarrow$ Image $32*32*3 $
    \begin{itemize}
        \item Convolution 2D
        \item Batch Normalization 2D
        \item ReLU
    \end{itemize}
\end{block}

\column{0.5\textwidth}
\begin{figure}
    \centering
    \includegraphics[width=0.8\linewidth]{img/decoder.png}
    \caption{Décodeur convolutionel}
    \label{fig:decoder}
\end{figure}
\end{columns}
\begin{block}{}
    Opération inverse de l'encodeur $\longrightarrow$ reconstruire l'image à partir de l'espace latent.
\end{block}
\end{frame}

\begin{frame}{Architecture du modèle Autoencodeur}
   
\begin{columns}
\column{0.5\textwidth}

\begin{block}{Classification}
    Espace latent $\Longrightarrow$ Classe de l'image
    \begin{itemize}
        \item Flatten
        \item Fully Connected Layer
        \item Relu
    \end{itemize}
\end{block}

\column{0.5\textwidth}
\begin{figure}
    \centering
    \includegraphics[width=0.8\linewidth]{img/classifieur.png}
    \caption{Classifieur}
    \label{fig:classifieur}
\end{figure}
\end{columns}

\vspace{20 pt}

\begin{block}{}
    Prédire la classe parmis les 10 classes de CIFAR-10.
\end{block}
\end{frame}

\begin{frame}{Architecture du modèle Autoencodeur}

    
    \vspace{10 pt}
    {\Large Métriques d'évaluation:}

    \vspace{20 pt}

    \begin{columns}
    \column{0.3\textwidth}
    \begin{block}{Décodeur}
        Loss: \textbf{MSE}
        Validation: \textit{MSE}
    \end{block}
    \column{0.3\textwidth}
    \begin{block}{Classifieur}
        Loss: \textit{Cross-Entropy}
        Validation: \textbf{Accuracy}
        
    \end{block}
    \column{0.3\textwidth}
    \begin{block}{Encodeur}
        Loss : \textit{custom loss}
        Validation : \textit{custom}
    \end{block}
    \end{columns}

    \begin{block}{}

        Loss custom = $\alpha$ * Loss du décodeur + $\beta$ * Loss du classifieur
    
    \end{block}

    \vspace{50 pt}

    \text{loss}: annoncée dans l'article\\
    \textit{loss}: déduite
\end{frame}


\subsection{Environnement de simulation}

\begin{frame}{Environnement de simulation}
    \begin{block}{}
        Simuler la communication entre les capteurs et l'UAV 
    \end{block}
    \begin{columns}
    \column{0.5\textwidth}

    \vspace{40 pt}
    \begin{block}{Simulation d'agrégation}
        \begin{itemize}
            \item grille de 300 * 300
            \item 4 sensors 
            \item 1 UAV
            \item Framework \textbf{GrADyS-SIM NG}
        \end{itemize}
    \end{block}
    \column{0.5\textwidth}
        \begin{figure}
        \centering
        \includegraphics[width=1\textwidth]{img/gridvoid.png}
        \end{figure}
    \end{columns}

\end{frame}


\begin{frame}{Environnement de simulation}
    \begin{figure}
        \centering
        \only<1>{\includegraphics[width=0.6\textwidth]{img/drone1.png}}%
        \only<2>{\includegraphics[width=0.62\textwidth]{img/drone2.png}}%
        \only<3>{\includegraphics[width=0.6\textwidth]{img/drone3.png}}%
        \only<4>{\includegraphics[width=0.6\textwidth]{img/drone4.png}}%
    \end{figure}
\end{frame}

\begin{frame}{Répartition des données}
    \begin{block}{CIFAR-10}
        Images de 32*32*3 $\|$ 10 classes $\|$ 6000 images par classe
    \end{block}

    \begin{figure}
        \centering
        \includegraphics[width=0.8\textwidth]{img/repartition.png}
    \end{figure}

    \begin{block}{}
        Simuler des données non-IID (non indépendantes et identiquement distribuées) entre les capteurs.
    \end{block}
    
\end{frame}

\begin{frame}{Aggrégation des modèles locaux}

    \begin{block}<1->{Entrainement local (capteurs)}
        Entrainement sur le dataset local durant $N$ epochs
    \end{block}

    \begin{block}<2->{Requête de communication (capteurs)}
        À $N$ epochs on stoppe l'entrainement et attend l'UAV
    \end{block}

    \begin{block}<3->{Transmission des poids locaux (capteurs $\rightarrow$ UAV)}
        On envoie envoie une version compressée des poids à l'UAV
    \end{block}

    \begin{block}<4->{Aggrégation des poids des modèles locaux (UAV)}
        Moyenne des poids des sensors qui ont communiqué avec l'UAV
    \end{block}

    \begin{block}<5->{Distribution des nouveaux poids globaux (UAV $\rightarrow$ capteurs)}
        L'UAV distribue les nouveaux poids globaux aux capteurs
    \end{block}

\end{frame}

\begin{frame}[c]{Optimisation des communications}

    \begin{columns}
    \column{0.5\textwidth}
    \begin{enumerate}
        \item <1-> Entrainement local (QAT)
        \item <2-> Quantification (-75\%)
        \item <3-> Compression (gzip)
        \item <4-> Décompression (lossless)
        \item <5-> Aggrégation
        \item <6-> Distribution
        \item <7-> Mise à jour
    \end{enumerate}
    \column{0.5\textwidth}
        \begin{figure}

            \centering
            \only<1>{\includegraphics[width=0.8\textwidth]{img/send0.png}}%
            \only<2>{\includegraphics[width=0.8\textwidth]{img/send1.png}}%
            \only<3>{\includegraphics[width=0.8\textwidth]{img/send2.png}}%
            \only<4>{\includegraphics[width=0.8\textwidth]{img/send3.png}}%
            \only<5>{\includegraphics[width=0.8\textwidth]{img/send4.png}}%
            \only<6>{\includegraphics[width=0.8\textwidth]{img/send5.png}}%
            \only<7>{\includegraphics[width=0.8\textwidth]{img/send0.png}}%
        \end{figure}
    \end{columns}
    
\end{frame}
